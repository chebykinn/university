\documentclass[a4paper, 12pt]{article}
\usepackage{geometry}
\geometry{verbose,a4paper,tmargin=2cm,bmargin=2cm,lmargin=2cm,rmargin=2cm}
\usepackage[T2A]{fontenc}
\usepackage[utf8]{inputenc}
\usepackage[english,russian]{babel}
\newif\ifisinsp
\newif\ifisone
\newif\ifisname
\isinsptrue
\isonetrue
\isnamefalse
% Это для нумерации страниц после титульника
\usepackage{fancyhdr}
\pagestyle{fancy}
\renewcommand{\headrulewidth}{0pt}
\fancyfoot[C] {\thepage}

\isonefalse
\def \labnum {1}
\def \labsubj {Тестирование программного обеспечения}
\def \labauthor {Айтуганов Д. А. \\ Чебыкин И. Б.}
\def \labgroup {P3301}
\isinspfalse
\def \labinsp {}
\def \labname {Вариант: 756}
\isnametrue

\usepackage{graphicx}
\usepackage{verbatim}
\usepackage[dvipsnames]{xcolor}

\usepackage{fancyvrb}

\RecustomVerbatimCommand{\VerbatimInput}{VerbatimInput} {
 fontsize=\scriptsize,
 %
 frame=lines,  % top and bottom rule only
 framesep=2em, % separation between frame and text
 rulecolor=\color{Gray},
 %
 label=\fbox{\color{Black}source},
 labelposition=topline,
 %
}

\begin{document}
\begin{titlepage}
	\begin{center}
		\large
		Университет ИТМО

		\vspace{0.25cm}
		
		Факультет программной инженерии и компьютерной техники
		
		Кафедра вычислительной техники
		\vfill
		
		\textsc{Лабораторная работа  № \labnum{} по дисциплине \\"\labsubj" \ifisname\small \\ \labname \fi}
			
		\bigskip
	\end{center}
	\vfill
	\vfill
	
	\begin{flushright}
	\ifisone
	Выполнил: \labauthor
	\else
	Выполнили: \labauthor
	\fi

	\vspace{0.25cm}
	Группа: \labgroup
			
	\vspace{0.25cm}
	\ifisinsp
	Проверяющий: \labinsp
	\fi
	\end{flushright}
	\vfill
	
	\begin{center}
	СПб, \the\year
	\end{center}
\end{titlepage}
\section{Задание}
\begin{enumerate}
\item Для указанной функции провести модульное тестирование разложения функции в степенной ряд. Выбрать достаточное тестовое покрытие.

\item Провести модульное тестирование указанного алгоритма. Для этого выбрать характерные точки внутри алгоритма, и для предложенных самостоятельно наборов исходных данных записать последовательность
попадания в характерные точки. Сравнить последовательность попадания с эталонной.  \item Сформировать доменную модель для заданного текста. Разработать тестовое покрытие для данной до-
менной модели.
\end{enumerate}

\section{Выполнение}

\subsection{Функция cos(x)}
Для тестирования для были выделены точки, в которых функция меняется одинаково:
\begin{enumerate}
\item $(-\pi, \pi)$
\item Значения на границах $-\pi, \pi$
\item Значения за границами $(-\infty, \pi), (\pi, +\infty)$
\end{enumerate}

\subsection{Расширяющееся дерево}
Данное дерево, основанное на бинарном, позволяет быстро осуществлять доступ
к недавно добавленным элементам.

Это реализуется за счет перемещения вершины в корень.
Таким образом можно рассмотреть три случая:
\begin{itemize}
\item Если родительская вершина -- корень, тогда достаточно поменять их местами.
\item Если родительская вершина не корень, и если она и текущая вершина являются
левыми или правыми сыновями, тогда сначала меняются местами родитель и родитель
родителя, а затем текущая вершина меняется с родителем, становясь корнем.
\item Если родительская вершина и текущая не одинаково левые или правые сыновья,
тогда они меняются местами, затем текущая вершина меняется с родителем родительской
вершины.

\end{itemize}
\section{Описание предметной области}
Медленно и осторожно он подошел к первому телу. Оно лежало обнадеживающе тихо,
и продолжало лежать так, когда он приблизился к нему вплотную и поставил ногу
на килобац, который оно все еще сжимало скрюченными пальцами.
\section{UML диаграмма}
\includegraphics[width=400bp]{img/uml.png}
\section{Вывод}
В ходе выполнения данной лабораторной работы было проведено тестирование
разработанных программых модулей с использованием средств JUnit4. Данная библиотека
предоставляет удобные средства тестирования и, в отличие от JUnit 3 она использует
такие удобные средства языка как аннотации.

\end{document}
